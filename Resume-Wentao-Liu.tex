\documentclass[margin, 10pt]{res} % Use the res.cls style, the font size can be changed to 11pt or 12pt here
%\usepackage{helvet} % Default font is the helvetica postscript font
%\usepackage{newcent} % To change the default font to the new century schoolbook postscript font uncomment this line and comment the one above
%\usepackage{times}

\usepackage{palatino}
%\linespread{1.05} % Palatino needs more leading (space between lines)

\usepackage{hyperref}
\usepackage{marvosym}
\usepackage{xeCJK}

\setCJKmainfont{SourceHanSansSC-Normal}
\setmainfont{Palatino}

\hypersetup{colorlinks, breaklinks, linkcolor=black, urlcolor=blue, anchorcolor=black, citecolor=black}
\setlength{\textwidth}{5.1in} % Text width of the document

\begin{document}

\def \Email {wentao.liu.zero@gmail.com}
\def \Name {WENTAO LIU}
\def \Phone {(+86)182-0026-0891 }
\def \Address {Sichuan University, Wangjiang Campus, Chengdu, China 610065}

\def \HEADER {
	\moveleft.5\hoffset\centerline{\LARGE \bf \Name} % Your name at the top
	\moveleft\hoffset\vbox{\hrule width\resumewidth height 1pt}\smallskip % Horizontal line after name; adjust line thickness by changing the '1pt'
	\moveleft.5\hoffset\centerline{\Address}
	\moveleft.5\hoffset\centerline{\Phone \href{mailto:\Email}{\underline{\smash{\Email}}}}}

\def \SICPlink {https://github.com/WentaoZero/SICP-Solution}
\def \DesignOfComputerProgramsLink {https://github.com/WentaoZero/Design-of-Computer-Programs}
\def \UnderstandingComputationLink {https://github.com/WentaoZero/Understanding-Computation}
\def \AlgorithmsUnlockedLink {https://github.com/WentaoZero/Algorithms-Unlocked}
\def \IntroToDataScienceLink {https://github.com/WentaoZero/Intro-to-Data-Science}
\def \IntroToParallelComputingLink {https://github.com/WentaoZero/Intro-to-Parallel-Programming}
\def \IntroToAlgorithmsLink {https://github.com/WentaoZero/Intro-to-Algorithms}

\def \VindentProj {\vspace*{-1.5em}}
\def \VindentMedia {\vspace*{-0.7em}}

\def \OBJECTIVE {OBJECTIVE}
\def \EDUCATION {EDUCATION}
\def \MaterEDU {
	{\sl M.S.}, Computer Science
	\hspace*{\fill}
	Sept. 2015 - expected Jun. 2018\\
	School of Computer Science
	\hspace*{\fill}
	Sichuan University
	}

\def \BachelorEDU {
	{\sl B.E.}, Software Engineering
	\hspace*{\fill}
	Sept. 2011 - Jun. 2015\\
	School of Software Engineering
	\hspace*{\fill}
	Sichuan University
	}

\def \KNOWLEDGE {KNOWLEDGE}
\def \DesignOfPrograms {Design of Programs}
\def \SICPsolution {{\it Structure and Interpretation of Computer Programs} Solution}
\def \DesignOfComputerPrograms {Design of Computer Programs, Udacity CS212}
\def \Algorithm {Algorithm}
\def \AlgorithmsUnlocked {{\it Algorithms Unlocked} Programs}
\def \IntroToAlgorithms {Intro to Algorithms, Udacity CS215}
\def \Other {Other}
\def \UnderstandingComputation {{\it Understanding Computation} Programs}
\def \IntroToDataScience {Intro to Data Science, Udacity UD359}


\def \MEDIA {MEDIA}
\def \LANGUAGE {LANGUAGE}
\def \Chinese {
		Chinese
		\hspace*{\fill}
		Native
	}

\def \English {
	English
	\hspace*{\fill}
	TOEFL iBT 90 (R 28, L 22, S 20, W 20)
	}

\def \RESUME
{
\HEADER

\begin{resume}

\section{\EDUCATION}

\MaterEDU
\\

\VindentProj
\BachelorEDU

\section{\KNOWLEDGE}

{\bf \DesignOfPrograms}
\\
\SICPsolution
\\
(Racket)
\hspace*{\fill}
\href{\SICPlink}{\underline{\smash{\SICPlink}}}
\\

\VindentProj
\DesignOfComputerPrograms
\\
(Python)
\hspace*{\fill}
\href{\DesignOfComputerProgramsLink}{\underline{\smash{\DesignOfComputerProgramsLink}}}\\

{\bf \Algorithm}
\\
\AlgorithmsUnlocked
\\
(Python)
\hspace*{\fill}
\href{\AlgorithmsUnlockedLink}{\underline{\smash{\AlgorithmsUnlockedLink}}}\\

\VindentProj
\IntroToAlgorithms
\\
(Python)
\hspace*{\fill}
\href{\IntroToAlgorithmsLink}{\underline{\smash{\IntroToAlgorithmsLink}}}\\

{\bf \Other}
\\
\UnderstandingComputation
\\
(Ruby)
\hspace*{\fill}
\href{\UnderstandingComputationLink}{\underline{\smash{\UnderstandingComputationLink}}}
\\

\VindentProj
\IntroToDataScience
\\
(numpy, pandas, ggplot)
\hspace*{\fill}
\href{\IntroToDataScienceLink}{\underline{\smash{\IntroToDataScienceLink}}}\\


\section{\MEDIA}
\def \GitHub {https://github.com/WentaoZero}
\def \StackOverflow {http://stackoverflow.com/users/3853711/Wentao}
\def \LinkedIn {https://www.linkedin.com/in/WentaoZero}


GitHub
\hspace*{\fill}
\href{\GitHub}{\underline{\smash{\GitHub}}}

\VindentMedia
StackOverflow
\hspace*{\fill}
\href{\StackOverflow}{\underline{\smash{\StackOverflow}}}

\VindentMedia
LinkedIn
\hspace*{\fill}
\href{\LinkedIn}{\underline{\smash{\LinkedIn}}}

\section{\LANGUAGE}

\Chinese

\VindentMedia

\English

\end{resume}
}

\RESUME

\newpage

\def \Name {刘文韬}
\def \Address {成都市,武侯区,四川大学,望江校区,610065}
\def \OBJECTIVE {求职意向}
\def \EDUCATION {教育背景}

\def \MaterEDU {
	计算机科学硕士
	\hspace*{\fill}
	2015年9月-(预计)2018年6月\\
	计算机学院
	\hspace*{\fill}
	四川大学}

\def \BachelorEDU {
	软件工程学士
	\hspace*{\fill}
	2011年9月-2015年6月\\
	软件学院
	\hspace*{\fill}
	四川大学}

\def \KNOWLEDGE {专业知识}

\def \DesignOfPrograms {程序设计}
\def \SICPsolution {《计算机程序的构造与解释》习题}
%\def \DesignOfComputerPrograms {Design of Computer Programs,优达学城 CS212}
\def \Algorithm {算法}
\def \AlgorithmsUnlocked {《算法基础:打开算法之门》习题}
%\def \IntroToAlgorithms {Intro to Algorithms,优达学城 CS215}
\def \Other {其他}
\def \UnderstandingComputation {《计算的本质:深入剖析程序和计算机》习题}
%\def \IntroToDataScience {Intro to Data Science,优达学城 UD359}

\def \MEDIA {网络社区}
\def \LANGUAGE {语言能力}
\def \Chinese {
		汉语
		\hspace*{\fill}
		母语
	}

\def \English {
	英语
	\hspace*{\fill}
	托福 iBT 90 (读 28,听 22,说 20,写 20)
	}

\RESUME

\end{document}